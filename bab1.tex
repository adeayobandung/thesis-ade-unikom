%-----------------------------------------------------------------------------%
\chapter{\babSatu}
%-----------------------------------------------------------------------------%
%\todo{tambahkan kata-kata pengantar bab 1 disini}


%-----------------------------------------------------------------------------%
\section{Latar Belakang}
%-----------------------------------------------------------------------------%
%\todo{tuliskan latar belakang penelitian disini}
Perkembangan dunia bisnis terus bersaing menghadapi era globalisasi, saling menunjukkan untuk berbagai kebutuhan konsumen yang semakin tinggi. Mulai dari kalangan menengah hingga kalangan menengah atas pasti menuntut kualitas terbaik dan harga ekonomis. Perubahan ekonomi yang cukup signifikan, apalagi negara yang sedang berkembang seperti di Indonesia, yang semakin hari semakin mengalami peningkatan di bidang ekonomi maupun industri.

Bisnis di Indonesia yang bergerak di bidang industri jasa maupun manufaktur pada umumnya bertujuan untuk mendapatkan laba yang maksimal dan menekan pengeluaran agar perusahaan tetap kompetitif. Salah satu faktor yang mengeluarkan banyak biaya dalam mendasari produk yaitu adanya management dalam mendapatkan produk, peramalan kebutuhan, pengadaan material, pengendalian persediaan, penyimpanan, distribusi/transfortasi ke distributor dan retail.

Banyaknya persaingan yang teradapat dalam suatu bisnis menjadi sebuah tantangan dalam menghadapi era globalisasi yang di hadapi para pebisnis di Indonesia. Dengan melibatkan banyak pihak juga merupakan suatu tuntutan dalam memenuhi kebutuhan konsumen agar lebih baik lagi seperti adanya suatu kebijakan agar perusahaan memiliki legalitas izin usaha yang dapat menambah kepercayaan penuh kepada konsumen.

Jika dilihat secara mendalam, ternyata esensi dari persaingan terletak pada bagaimana sebuah perusahaan dapat mengimplementasikan proses penciptaan produk dan/atau jasanya secara murah, lebih baik, dan lebih cepat \f{(cheaper, better, and faster)} dibandingkan dengan persaingan bisnisnya \cite{Manajemen}.

Karena ketatnya persaingan dan perubahan lingkungan bisnis akhir-akhir ini menuntut adanya sebuah model baru dalam pengelolaan aliran produk/informasi terutama pada pemasaran produk, yang merupakan modifikasi dari metode sebelumnya (manajemen logistik), yaitu \SCM (\scm).

Dengan adanya \SCM yang berikut akan terus dikatakan \scm merupakan suatu sistem yang mengelola dengan sistem yang lainnya yang bertujuan untuk memenuhi kebutuhan pelanggan. \scm merupakan rangkaian kegiatan perencanaan, koordinasi, dan pengendalian seluruh proses bisnis dan aktifitas dalam \f{supply chain} untuk menciptakan \f{customer value} terbaik dengan biaya efisien namun tetap memenuhi seluruh kebutuhan \f{stakeholder} lain dalam \f{supply chain} \cite{Ibrahim, Hilman2013}. Proses pemenuhan kebutuhan pelanggan ini di mulai dari bahan mentah hingga sampai ke pelanggan. Umumnya gambaran sistem \scm hanyalah \f{supplier}, distributor dan \f{customer}, akan tetapi pada kenyataannya sistem \scm akan melibatkan banyak pihak. \scm juga akan mengaplikasikan bagaimana mengaplikasi suatu jaringan pada kegiatan produksi dan distribusi dari suatu perusahaan dapat bekerja bersama-sama agar dapat menehui kebutuhan konsumen.

Chaca Komputer merupakan salah satu toko komputer yang berlokasi di Telana Pura Kota Jambi, Chaca Komputer ini memiliki proses bisnis di mulai dari melakukan service dan menjual berbagai macam suku cadang untuk komputer. Berdasarkan hasil pengamatan wawancara dengan pihak toko bahwa sistem informasi penjualan seringkali mengalami ketidak sesuain stok, adapun persediaan suku cadang ini berdasarkan pembelian yang di pesan melalui pedagang suku cadang komputer yang di kirim langsung ke toko chaca komputer. Terkadang juga setelah dilakukan perhitungan oleh pegawai ketika melakukan perhitungan stok barang terkadang tidak cocok dengan barang yang tersimpan  sehingga mengalami ketidak sesuaian dengan persediaan yang telah di tentukan, hal tersebut akan mempengaruhi ketersediaan kebutuhan konsumen yang tidak dapat di prediksi dan berdampak pada kerugian terhadap toko.

Oleh karena itu untuk mengatasi permasalahan di atas, penulis menuangkan ide untuk merancang sebuah sebuah sistem yang didukung dengan metode penunjang yang dipilih dalam pengelolaan proses persediaan bahan di Toko Chaca Komputer untuk memastikan persediaan dapat memenuhi kebutuhan yang ada. Salah satu metode pengelolaan rantai persediaan (\SCM). Konsep \scm merupakan mekanisme proses pemenuhan kebutuhan pelanggan mulai dari bahan mentah hinggal sampai ke konsumen. \scm merupakan mekanisme untuk meningkatkan produktivitas total perusahaan dalam rantai suplai melalui optimalisasi waktu, lokasi, dan aliran bahan. Dengan \scm, waktu pemesanan akan lebih teratur setiap kali periode pemesanan, dan keadaan persediaan yang akan habis lebih mudah diketahui \cite{Ibrahim}.


%-----------------------------------------------------------------------------%
\section{Rumusan Masalah}
%-----------------------------------------------------------------------------%
Berdasarkan latar belakang tersebut maka rumusan masalahnya adalah bagaimana
manajemen rantai pasok (\SCM) di Toko Chaca Komputer untuk menjaga keberlangsungan bisnis.

%-----------------------------------------------------------------------------%
\section{Batasan Masalah}
%-----------------------------------------------------------------------------%
Agar penelitian ini dapat terarah dan jelas dalam ruang lingkup penelitiannya, maka penelitian ini dibatasi pada :
\begin{enumerate}
	\item Penelitian ini hanya mengenai manajemen rantai pasokan sebagai variabel independennya (bebas) dengan dimensi koordinasi, perencanaan, keterlibatan pemasok, dan keterlibatan konsumen.
	
	\item Penelitian ini berfokus peningkatkan produktivitas dan efisiensi rantai pasok persediaan barang.
	
	\item Pendekatan \f{Supply Chain Management} (SCM) dikhususkan untuk mengintegrasikan proses bisnis dan meminimisasi biaya.
	
	\item Responden pada penelitian ini adalah karyawan bagian operasi Toko Chaca Komputer. 
\end{enumerate}


%-----------------------------------------------------------------------------%
\section{Tujuan Penelitian}
%-----------------------------------------------------------------------------%
Tujuan dari penelitian ini adalah untuk mengetahui manajemen rantai pasok terhadap persediaan barang di Toko Chaca Komputer Jambi.

%-----------------------------------------------------------------------------%
\section{Manfaat Penelitian}
%-----------------------------------------------------------------------------%
Hasil dari penelitian ini diharapkan dapat memberi manfaat, di antaranya adalah sebagai berikut:
\begin{enumerate}
	\item Bagi Perusahaan
	
	Penelitian ini diharapkan dapat bermanfaat pada bagian operasi perusahaan, yang nantinya akan dijadikan sebagai informasi dan masukan untuk mengetahui langkah – langkah atau kebijakan yang akan diambil dalam meningkatkan kinerja dan melakukan suatu kebijakan untuk menambah dan mempertahankan keunggulan kompetitifnya.
	
	\item Bagi pembaca
	
	Sebagai tambahan ilmu dan informasi terbaru bagi pembaca dan mahasiswa lainnya yang memerlukan. 
	\item Bagi Penulis 
	
	Untuk mendapatkan gambaran dari perumusan masalah diatas dan mengaplikasikan ilmu yang telah didapat selama perkuliahan 
\end{enumerate}

%-----------------------------------------------------------------------------%
\section{Lokasi dan Waktu Penelitian}
%-----------------------------------------------------------------------------%
Berikut adalah lokasi dan waktu penelitian dari penelitian yang penulis lakukan adalah sebagai berikut:

\subsection{Lokasi Penelitian}
Lokasi penelitian yang dilakukan oleh penulis yaitu di Toko Chaca Komputer di Telanai Pura Kota Jambi.

\subsection{Waktu Penelitian}
Adapun waktu penelitian yang akan di laksanakan yang di tampilkan dalam \tab~\ref{tab:tab1} berikut :

\begin{table}[H]
	\centering
	\caption{Waktu Penelitian}
	\label{tab:tab1}
	\begin{tabular}{|c|c|c|c|c|c|}
		\hline
		\multirow{2}{*}{\bo{No.}} & \multirow{2}{*}{\bo{Nama Kegiatan}} & \multicolumn{4}{|c|}{\bo{Bulan}}\\
		\cline{3-6} & & Minggu 1 & Minggu 2 & Minggu 3 & Minggu 4\\
		\hline
		1 & Kegiatan 1 & & & &\\
		\hline
		2 & Kegiatan 2 & & & &\\
		\hline
		3 & Kegiatan 3 & & & &\\
		\hline
		4 & Kegiatan 4 & & & &\\
		\hline
	\end{tabular}
\end{table}
	
Berikut keterangan dari nama kegiata pada tabel:
\begin{enumerate}
	\item \bo{Kegiatan 1} : Melakukan pengumpulan kebutuhan informasi di lokasi
	penelitian berupa data penerimaan mahasiswa baru
	\item \bo{Kegiatan 2} : Melakukan Analisa data berdasarkan data yang telah di
	dapatkan
	\item \bo{Kegiatan 3} : Melakukan implementasi model dan pengkodean.
	\item \bo{Kegiatan 4} : Melakukan evaluasi data dari hasil implementasi
	\item \bo{Kegiatan 5} : Membuat kesimpulan hasil dan saran
\end{enumerate}



%-----------------------------------------------------------------------------%
\section{Sistematika Penulisan}
%-----------------------------------------------------------------------------%
Sistematika penulisan laporan adalah sebagai berikut:
\begin{itemize}
	\item \bo{BAB 1 \babSatu} \\
	Dalam bab ini memuat uraian mengenai latar belakang, Rumusan Masalah, Batasan Masalah, Tujuan dan Manfaat Penelitian, dan Sistematika Penulisan. 
	
	\item \bo{BAB 2 \babDua} \\
	Bab ini berisikan tinjauan pustaka yang menjadi bahan acuan penyusunan skripsi yang mempunyai relevansi dengan pembahasan yang dilakukan. Bab ini juga berisikan rerangka pemikiran dan perumusan hipotesa. 
	
	\item \bo{BAB 3 \babTiga} \\ 
	Bab ini berisikan uraian tentang rancangan penelitian, variabel dan pengukuran, definisi operasional variabel, sampel dan pengumpulan data, uji instrumen serta metode analisis data yang digunakan dalam penelitian ini.
	
	\item \bo{BAB 4 \babEmpat} \\
	Bab ini berisikan analisis statistik deskriptif dari tiap-tiap variabel yang menunjang pembahasan hasil penelitian. Uraian selanjutnya adalah hasil penelitian yang menguji kesesuaian model dan pengujian hipotesa kemudian pembahasan akhir.
	
	\item \bo{BAB 5 \babLima} \\ 
	Bab ini merupakan akhir dari skripsi yang berisikan simpulan dari pembahasan pada bab – bab sebelumnya. Implikasi manajerial, keterbatasan penelitian, serta saran untuk penelitian selanjutnya. 
\end{itemize} 

