%-----------------------------------------------------------------------------%
\chapter{\babSatu}
%-----------------------------------------------------------------------------%
%\todo{tambahkan kata-kata pengantar bab 1 disini}


%-----------------------------------------------------------------------------%
\section{Latar Belakang}
%-----------------------------------------------------------------------------%
%\todo{tuliskan latar belakang penelitian disini}
Perkembangan dunia bisnis terus bersaing menghadapi era globalisasi, saling menunjukkan untuk berbagai kebutuhan konsumen yang semakin tinggi. Mulai dari kalangan menengah hingga kalangan menengah atas pasti menuntut kualitas terbaik dan harga ekonomis. Perubahan ekonomi yang cukup signifikan, apalagi negara yang sedang berkembang seperti di Indonesia, yang semakin hari semakin mengalami peningkatan di bidang ekonomi maupun industri.

Bisnis di Indonesia yang bergerak di bidang industri jasa maupun manufaktur pada umumnya bertujuan untuk mendapatkan laba yang maksimal dan menekan pengeluaran agar perusahaan tetap kompetitif. Salah satu faktor yang mengeluarkan banyak biaya dalam mendasari produk yaitu adanya management dalam mendapatkan produk, peramalan kebutuhan, pengadaan material, pengendalian persediaan, penyimpanan, distribusi/transfortasi ke distributor dan retail.

Jika dilihat secara mendalam, ternyata esensi dari persaingan terletak pada bagaimana sebuah perusahaan dapat mengimplementasikan proses penciptaan produk dan/atau jasanya secara murah, lebih baik, dan lebih cepat \f{(cheaper, better, and faster)} dibandingkan dengan persaingan bisnisnya \cite{Manajemen}.

Karena ketatnya persaingan dan perubahan lingkungan bisnis akhir-akhir ini menuntut adanya sebuah model baru dalam pengelolaan aliran produk/informasi terutama pada pemasaran produk, yang merupakan modifikasi dari metode sebelumnya (manajemen logistik), yaitu \SCM (\scm).



%-----------------------------------------------------------------------------%
\section{Rumusan Masalahan}
%-----------------------------------------------------------------------------%
Pada bagian ini akan dijelaskan mengenai definisi permasalahan 
yang \saya~hadapi dan ingin diselesaikan serta asumsi dan batasan 
yang digunakan dalam menyelesaikannya.


%-----------------------------------------------------------------------------%
\subsection{Definisi Permasalahan}
%-----------------------------------------------------------------------------%
\todo{Tuliskan permasalahan yang ingin diselesaikan. Bisa juga
	berbentuk pertanyaan}


%-----------------------------------------------------------------------------%
\subsection{Batasan Permasalahan}
%-----------------------------------------------------------------------------%
\todo{Umumnya ada asumsi atau batasan yang digunakan untuk 
	menjawab pertanyaan-pertanyaan penelitian diatas.}


%-----------------------------------------------------------------------------%
\section{Tujuan}
%-----------------------------------------------------------------------------%
\todo{Tuliskan tujuan penelitian.}


%-----------------------------------------------------------------------------%
\section{Posisi Penelitian}
%-----------------------------------------------------------------------------%
\todo{Posisi penelitian Anda jika dilihat secara bersamaan dengan 
	peneliti-peneliti lainnya. Akan lebih baik lagi jika ikut menyertakan 
	diagram yang menjelaskan hubungan dan keterkaitan antar 
	penelitian-penelitian sebelumnya}


%-----------------------------------------------------------------------------%
\section{Metodologi Penelitian}
%-----------------------------------------------------------------------------%
\todo{Tuliskan metodologi penelitian yang digunakan.}


%-----------------------------------------------------------------------------%
\section{Sistematika Penulisan}
%-----------------------------------------------------------------------------%
Sistematika penulisan laporan adalah sebagai berikut:
\begin{itemize}
	\item Bab 1 \babSatu \\
	\item Bab 2 \babDua \\
	\item Bab 3 \babTiga \\
	\item Bab 4 \babEmpat \\
	\item Bab 5 \babLima \\
	\item Bab 6 \babEnam \\
	\item Bab 7 \kesimpulan \\
\end{itemize}

\todo{Tambahkan penjelasan singkat mengenai isi masing-masing bab.}

